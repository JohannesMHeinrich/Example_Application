% ----------------------------------------------------------------------------------------
% ---  27/10/2019 johannes.m.heinrich@gmail.com ---------------------------
% ----------------------------------------------------------------------------------------

\documentclass[a4paper]{class_cv}

\usepackage{tikz}
\usetikzlibrary{arrows.meta}

%\usepackage{palatino}
\usepackage[sfdefault]{cabin}
%\usepackage[default]{raleway}
%\usepackage{lmodern}



%----------------------------------------------------------------------------------------
%	 PERSONAL INFORMATION
%----------------------------------------------------------------------------------------

% Profile picture
\profilepic{} 

% Your name
\cvname{Johannes}

% Your name
%\cvfullname{Dr Johannes Matthias Heinrich} 

% Job title/career
\cvjobtitle{Research Associate} 

% Date of birth
\cvdate{22/11/1985, Augsburg, DE} 

% Short address/location, use \newline if more than 1 line is required
\cvaddress{Dr Johannes Matthias Heinrich\newline 72 Burton Rd,\newline Camberwell, London,\newline SW9 6TQ} 

% Phone number
\cvnumberphone{+49 1704010308}

% Homepage
\cvsite{}

% Email address
\cvmail{johannes.m.heinrich@gmail.com} 

%----------------------------------------------------------------------------------------




%----------------------------------------------------------------------------------------
%	 Locations for the boxes in the cv timeline
%----------------------------------------------------------------------------------------

\newcommand\circlesize{0.2}
\newcommand\circlesizetwo{0.1}

\newcommand\xtimeline{3.}
\newcommand\xdates{0.15}
\newcommand\xwidth{12.4}

\newcommand\yholbeinstart{0.0}
\newcommand\yholbeinstop{3.15}

\newcommand\ywuestart{3.85}
\newcommand\ywuestop{7.1}

\newcommand\ysglstart{7.85}
\newcommand\ysglstop{10.55}

\newcommand\yulmstart{11.3}
\newcommand\yulmstop{13.55}

\newcommand\yparisstart{14.3}
\newcommand\yparisstop{21.15}

\newcommand\londonstart{21.85}
\newcommand\londonstop{25.5}

\newcommand\offsetXA{1.}
\newcommand\offsetXB{.5}
\newcommand\offsetYA{.4}
\newcommand\offsetYB{.8}

\newcommand\deltaywue{0.626}

%----------------------------------------------------------------------------------------






\begin{document}

%----------------------------------------------------------------------------------------
%	 ABOUT ME
%----------------------------------------------------------------------------------------


\aboutme{

\vspace{0.1cm}
I am a physicist experienced in a variety of fields with a passion for Python and automation.

For the last five years I have been working on the design and implementation of state of the art experiments in atomic physics. The sequence of such experiments is controlled by a central computer, bundeling the control over the various devices. For this task, among others, I use Python.

My passion for soft- and hardware also expands to my private life, where I implement evolutionary algorithms for different problems and experiment with $\,$ hardware such as the raspberry pi.}

%----------------------------------------------------------------------------------------
%	 SKILLS
%----------------------------------------------------------------------------------------

\aboutmeskills{
\vspace{0.1cm}

Software:
\begin{itemize}
\item Python, pyqt
\item Mathematica
\item COMSOL multiphysics
\item SIMION
\item CATIA, SolidWorks
\item GitHub
\item Latex, Blender
\end{itemize}
\vspace{0.25cm}
\hrule
\vspace{0.1cm}
Hardware:
\begin{itemize}
\item ultra high vacuum creation
\item lasers and optics
\item electronics
\item raspberry pi, red pitaya
\end{itemize}
\vspace{0.25cm}
\hrule
\vspace{0.1cm}
Languages:
\begin{itemize}
\item English
\item German
\end{itemize}

}



%----------------------------------------------------------------------------------------
%	 Print sidebar
%----------------------------------------------------------------------------------------

\makeprofile



%----------------------------------------------------------------------------------------
%	 CV timeline
%----------------------------------------------------------------------------------------

\vspace{-0.3cm}

\profilesection{Curriculum vitae}

\vspace{0.5cm}

\begin{tikzpicture}

%	\draw[help lines,xstep=.5,ystep=.5, white] (0,0) grid (13,27);
	
	% ---------------------------------------------------------------------------------
	% --- TIMELINE ----------------------------------------------------------------	
	% ---------------------------------------------------------------------------------
	\draw [line width=0.5mm, mainblue, -{Latex[length=5mm, width=3.3mm]}] (\xtimeline, \yholbeinstart -0.25) -- (\xtimeline, \londonstop + 1.) ;
	

	% holbein -----------------------------------------------------------------------------
	% the box
	\draw [line width=0.2mm, mainblue]
	(\xtimeline - \offsetXA, \yholbeinstop) -- (\xwidth, \yholbeinstop) --
	(\xwidth, \yholbeinstart + \offsetYA) -- (\xtimeline, \yholbeinstart +\offsetYA) ;
	
	\draw [line width=0.2mm, mainblue ]
	(\xtimeline + \offsetXB, \yholbeinstop - \offsetYB) -- (\xwidth - \offsetXB, \yholbeinstop - \offsetYB) ;	
	
	% the circles
	\draw [mainblue,line width=0.5mm,fill=white] (\xtimeline,\yholbeinstop) circle (\circlesize) ;
	\draw [mainblue,line width=0.5mm,fill=mainblue] (\xtimeline,\yholbeinstop) circle (\circlesizetwo) ;

	% the dates
	\node[text width=1.75cm, anchor = west, inner sep=0pt, align=right] at (\xdates, \yholbeinstop) {05/2005};

	% the text
	\node[text width=8.5cm, anchor = north west,  inner sep=0pt, align=justify] at
	(\xtimeline + \offsetXB, \yholbeinstop - \offsetYA) 
	{{\textsc{Holbein Gymnasium Augsburg}}\\[0.25cm]
	Abitur with Award for Excellence in Physics by DPG\\[-0.3cm]
	\begin{itemize}
		\item advanced course mathematics: \hfill 14/15
		\item advanced course physics: \hfill 15/15
	\end{itemize}} ;

	% ---------------------------------------------------------------------------------------
			
	% wuerzburg -------------------------------------------------------------------------
	% the box
	\draw [line width=0.2mm, mainblue ]
	(\xtimeline - \offsetXA, \ywuestop) -- (\xwidth, \ywuestop) --
	(\xwidth, \ywuestart) -- (\xtimeline - \offsetXA, \ywuestart) ;
	
	\draw [line width=0.2mm, mainblue ]
	(\xtimeline + \offsetXB, \ywuestop - \offsetYB) -- (\xwidth - \offsetXB, \ywuestop - \offsetYB) ;	
	
	% the circles
	\draw [mainblue,line width=0.5mm,fill=mainblue] (\xtimeline,\ywuestart) circle (\circlesizetwo);
	\draw [mainblue,line width=0.5mm,fill=white] (\xtimeline,\ywuestop) circle (\circlesize);
	\draw [mainblue,line width=0.5mm,fill=mainblue] (\xtimeline,\ywuestop) circle (\circlesizetwo);

	% the dates
	\node[text width=1.75cm, anchor = west, inner sep=0pt, align=right] at (\xdates, \ywuestart) {10/2005};
	\node[text width=1.75cm, anchor = west, inner sep=0pt, align=right] at (\xdates, \ywuestop) {06/2013};

	% the text	
	\node[text width=8.5cm, anchor = north west, inner sep=0pt, align=justify] at 
	(\xtimeline + \offsetXB, \ywuestop - \offsetYA)
	{{\textsc{Julius-Maximilians-University W{\"u}rzburg}}\\[0.25cm]
	Study of physics with informatics and chemistry as minors, grade: gut\\[0.1cm]
	Diploma thesis with Prof. W. Hanke:\\[-0.3cm]
	\begin{center}Strong Coupling FRG Analysis of Frustrated Magnets \end{center}
	\vspace{0.25cm}};
	% ---------------------------------------------------------------------------------------
		
	% sgl ----------------------------------------------------------------------------------
	% the box
	\draw [line width=0.2mm, mainblue ]
	(\xtimeline - \offsetXA, \ysglstop) -- (\xwidth, \ysglstop) --
	(\xwidth, \ysglstart) -- (\xtimeline - \offsetXA, \ysglstart) ;
	
	\draw [line width=0.2mm, mainblue ]
	(\xtimeline + \offsetXB, \ysglstop - \offsetYB) -- (\xwidth - \offsetXB, \ysglstop - \offsetYB) ;	
	
	% the circles
	\draw [mainblue,line width=0.5mm,fill=mainblue] (\xtimeline,\ysglstart) circle (\circlesizetwo);
	\draw [mainblue,line width=0.5mm,fill=white] (\xtimeline,\ysglstop) circle (\circlesize);
	\draw [mainblue,line width=0.5mm,fill=mainblue] (\xtimeline,\ysglstop) circle (\circlesizetwo);
	
	% the dates
	\node[text width=1.75cm, anchor = west, inner sep=0pt, align=right] at (\xdates, \ysglstart) {10/2013};
	\node[text width=1.75cm, anchor = west, inner sep=0pt, align=right] at (\xdates, \ysglstop) {04/2014};

	% the text		
	\node[text width=8.5cm, anchor = north west,  inner sep=0pt, align=justify] at
	(\xtimeline + \offsetXB, \ysglstop - \offsetYA)
	{{\textsc{SGL Carbon, Department T\&I Modeling}}\\[0.25cm]
	Internship with Prof. T. Frommelt:\\[-0.3cm]
	\begin{center} Investigation of instabilities in the Hall-H\'{e}roult-process for smelting aluminium\end{center}
	};
	% ---------------------------------------------------------------------------------------
		
	% ulm ----------------------------------------------------------------------------------
	% the box
	\draw [line width=0.2mm, mainblue ]
	(\xtimeline - \offsetXA, \yulmstop) -- (\xwidth, \yulmstop) --
	(\xwidth, \yulmstart) -- (\xtimeline - \offsetXA, \yulmstart) ;
	
	\draw [line width=0.2mm, mainblue ]
	(\xtimeline + \offsetXB, \yulmstop - \offsetYB) -- (\xwidth - \offsetXB, \yulmstop - \offsetYB) ;	
	
	% the circles
	\draw [mainblue,line width=0.5mm,fill=mainblue] (\xtimeline,\yulmstart ) circle (\circlesizetwo);
	\draw [mainblue,line width=0.5mm,fill=white] (\xtimeline,\yulmstop ) circle (\circlesize);
	\draw [mainblue,line width=0.5mm,fill=mainblue] (\xtimeline,\yulmstop ) circle (\circlesizetwo);
	
	% the dates
	\node[text width=1.75cm, anchor = west, inner sep=0pt, align=right] at (\xdates, \yulmstart) {05/2014};
	\node[text width=1.75cm, anchor = west, inner sep=0pt, align=right] at (\xdates, \yulmstop) {09/2014};
	
	% the text		
	\node[text width=8.5cm, anchor = north west,  inner sep=0pt, align=justify] at
	(\xtimeline + \offsetXB, \yulmstop - \offsetYA)
	{{\textsc{Institute of Quantum Matter, University of Ulm}}\\[0.25cm]
	Internship with Prof. J. Hecker-Denschlag\\[-0.3cm]
	\begin{center} Work on a diode laser system at 671nm \end{center}};
	% ---------------------------------------------------------------------------------------
			
	% paris ----------------------------------------------------------------------------------
	% the box
	\draw [line width=0.2mm, mainblue ]
	(\xtimeline - \offsetXA, \yparisstop) -- (\xwidth, \yparisstop) --
	(\xwidth, \yparisstart) -- (\xtimeline - \offsetXA, \yparisstart) ;
	
	\draw [line width=0.2mm, mainblue ]
	(\xtimeline + \offsetXB, \yparisstop - \offsetYB) -- (\xwidth - \offsetXB, \yparisstop - \offsetYB) ;	
	
	% the circles
	\draw [mainblue,line width=0.5mm,fill=mainblue] (\xtimeline,\yparisstart ) circle (\circlesizetwo);
	\draw [mainblue,line width=0.5mm,fill=white] (\xtimeline,\yparisstop ) circle (\circlesize);	
	\draw [mainblue,line width=0.5mm,fill=mainblue] (\xtimeline,\yparisstop ) circle (\circlesizetwo);	
	
	% the dates
	\node[text width=1.75cm, anchor = west, inner sep=0pt, align=right] at (\xdates, \yparisstart) {10/2014};
	\node[text width=1.75cm, anchor = west, inner sep=0pt, align=right] at (\xdates, \yparisstop) {07/2018};
	
	% the text
	\node[text width=8.5cm, anchor = north west,  inner sep=0pt, align=justify] at
	(\xtimeline + \offsetXB, \yparisstop - \offsetYA)
	{{\textsc{Laboratoire Kastler Brossel, Sorbonne University}}\\[0.25cm]
	PostDoc with Prof. L. Hilico\\[-0.3cm]
	\begin{itemize}
		\item automation of sympathetically cooled ion detection via imaging by an EMCCD
%		\item a Python program to detect the amount of sympathetically cooled H$_{2}^{+}$ ions via imaging by an EMCCD.
	\end{itemize}
	\vspace{0.35cm}
	PhD thesis with Prof. L. Hilico:\\[-0.3cm]
	\begin{center}A Be$^{+}$ Ion Trap for H$_{2}^{+}$ Spectroscopy, \end{center}
	\vspace{0.15cm}
	a {\textsc{Marie Curie}} project funded by the European Union, which included design, assembly, and operation of\\[-0.3cm]
	\begin{itemize}
		\item a diode based master-slave laser system at 626nm
		\item an ion trap including supplies and vacuum system
		\item atomic ovens and imaging systems
		\item control software
	\end{itemize}};
	% ---------------------------------------------------------------------------------------
		
	% london ----------------------------------------------------------------------------------
	% the box
	\draw [line width=0.2mm, mainblue ]
	(\xtimeline - \offsetXA, \londonstop) -- (\xwidth, \londonstop) --
	(\xwidth, \londonstart) -- (\xtimeline - \offsetXA, \londonstart) ;
	
	\draw [line width=0.2mm, mainblue ]
	(\xtimeline + \offsetXB, \londonstop - \offsetYB) -- (\xwidth - \offsetXB, \londonstop - \offsetYB) ;	
	
	% the circles	
	\draw [mainblue,line width=0.5mm,fill=mainblue] (\xtimeline, \londonstart) circle (\circlesizetwo);	
	\draw [mainblue,line width=0.5mm,fill=white] (\xtimeline, \londonstop) circle (\circlesize);
	
	% the dates
	\node[text width=1.75cm, anchor = west, inner sep=0pt, align=right] at (\xdates, \londonstart) {10/2018};
	\node[text width=1.75cm, anchor = west, inner sep=0pt, align=right] at (\xdates, \londonstop) {11/2019};
	
	% the text
	\node[text width=8.5cm, anchor = north west,  inner sep=0pt, align=justify] at 
	(\xtimeline + \offsetXB, \londonstop - \offsetYA)
	{{\textsc{Imperial College London}}\\[0.25cm]
	Research Associate with Prof. R. Thompson\\[-0.3cm]
	\begin{itemize}
		\item implementation of a new ion trap experiment:\\
			   vacuum system, ovens, software, among others
		\item refurbishment of an established experiment
		\item co-supervision of master- and PhD students
	\end{itemize}};
	% ---------------------------------------------------------------------------------------
	
\end{tikzpicture}
   


\end{document} 
